
% Default to the notebook output style

    


% Inherit from the specified cell style.




    
\documentclass[11pt]{article}

    
    
    \usepackage[T1]{fontenc}
    % Nicer default font (+ math font) than Computer Modern for most use cases
    \usepackage{mathpazo}

    % Basic figure setup, for now with no caption control since it's done
    % automatically by Pandoc (which extracts ![](path) syntax from Markdown).
    \usepackage{graphicx}
    % We will generate all images so they have a width \maxwidth. This means
    % that they will get their normal width if they fit onto the page, but
    % are scaled down if they would overflow the margins.
    \makeatletter
    \def\maxwidth{\ifdim\Gin@nat@width>\linewidth\linewidth
    \else\Gin@nat@width\fi}
    \makeatother
    \let\Oldincludegraphics\includegraphics
    % Set max figure width to be 80% of text width, for now hardcoded.
    \renewcommand{\includegraphics}[1]{\Oldincludegraphics[width=.8\maxwidth]{#1}}
    % Ensure that by default, figures have no caption (until we provide a
    % proper Figure object with a Caption API and a way to capture that
    % in the conversion process - todo).
    \usepackage{caption}
    \DeclareCaptionLabelFormat{nolabel}{}
    \captionsetup{labelformat=nolabel}

    \usepackage{adjustbox} % Used to constrain images to a maximum size 
    \usepackage{xcolor} % Allow colors to be defined
    \usepackage{enumerate} % Needed for markdown enumerations to work
    \usepackage{geometry} % Used to adjust the document margins
    \usepackage{amsmath} % Equations
    \usepackage{amssymb} % Equations
    \usepackage{textcomp} % defines textquotesingle
    % Hack from http://tex.stackexchange.com/a/47451/13684:
    \AtBeginDocument{%
        \def\PYZsq{\textquotesingle}% Upright quotes in Pygmentized code
    }
    \usepackage{upquote} % Upright quotes for verbatim code
    \usepackage{eurosym} % defines \euro
    \usepackage[mathletters]{ucs} % Extended unicode (utf-8) support
    \usepackage[utf8x]{inputenc} % Allow utf-8 characters in the tex document
    \usepackage{fancyvrb} % verbatim replacement that allows latex
    \usepackage{grffile} % extends the file name processing of package graphics 
                         % to support a larger range 
    % The hyperref package gives us a pdf with properly built
    % internal navigation ('pdf bookmarks' for the table of contents,
    % internal cross-reference links, web links for URLs, etc.)
    \usepackage{hyperref}
    \usepackage{longtable} % longtable support required by pandoc >1.10
    \usepackage{booktabs}  % table support for pandoc > 1.12.2
    \usepackage[inline]{enumitem} % IRkernel/repr support (it uses the enumerate* environment)
    \usepackage[normalem]{ulem} % ulem is needed to support strikethroughs (\sout)
                                % normalem makes italics be italics, not underlines
    

    
    
    % Colors for the hyperref package
    \definecolor{urlcolor}{rgb}{0,.145,.698}
    \definecolor{linkcolor}{rgb}{.71,0.21,0.01}
    \definecolor{citecolor}{rgb}{.12,.54,.11}

    % ANSI colors
    \definecolor{ansi-black}{HTML}{3E424D}
    \definecolor{ansi-black-intense}{HTML}{282C36}
    \definecolor{ansi-red}{HTML}{E75C58}
    \definecolor{ansi-red-intense}{HTML}{B22B31}
    \definecolor{ansi-green}{HTML}{00A250}
    \definecolor{ansi-green-intense}{HTML}{007427}
    \definecolor{ansi-yellow}{HTML}{DDB62B}
    \definecolor{ansi-yellow-intense}{HTML}{B27D12}
    \definecolor{ansi-blue}{HTML}{208FFB}
    \definecolor{ansi-blue-intense}{HTML}{0065CA}
    \definecolor{ansi-magenta}{HTML}{D160C4}
    \definecolor{ansi-magenta-intense}{HTML}{A03196}
    \definecolor{ansi-cyan}{HTML}{60C6C8}
    \definecolor{ansi-cyan-intense}{HTML}{258F8F}
    \definecolor{ansi-white}{HTML}{C5C1B4}
    \definecolor{ansi-white-intense}{HTML}{A1A6B2}

    % commands and environments needed by pandoc snippets
    % extracted from the output of `pandoc -s`
    \providecommand{\tightlist}{%
      \setlength{\itemsep}{0pt}\setlength{\parskip}{0pt}}
    \DefineVerbatimEnvironment{Highlighting}{Verbatim}{commandchars=\\\{\}}
    % Add ',fontsize=\small' for more characters per line
    \newenvironment{Shaded}{}{}
    \newcommand{\KeywordTok}[1]{\textcolor[rgb]{0.00,0.44,0.13}{\textbf{{#1}}}}
    \newcommand{\DataTypeTok}[1]{\textcolor[rgb]{0.56,0.13,0.00}{{#1}}}
    \newcommand{\DecValTok}[1]{\textcolor[rgb]{0.25,0.63,0.44}{{#1}}}
    \newcommand{\BaseNTok}[1]{\textcolor[rgb]{0.25,0.63,0.44}{{#1}}}
    \newcommand{\FloatTok}[1]{\textcolor[rgb]{0.25,0.63,0.44}{{#1}}}
    \newcommand{\CharTok}[1]{\textcolor[rgb]{0.25,0.44,0.63}{{#1}}}
    \newcommand{\StringTok}[1]{\textcolor[rgb]{0.25,0.44,0.63}{{#1}}}
    \newcommand{\CommentTok}[1]{\textcolor[rgb]{0.38,0.63,0.69}{\textit{{#1}}}}
    \newcommand{\OtherTok}[1]{\textcolor[rgb]{0.00,0.44,0.13}{{#1}}}
    \newcommand{\AlertTok}[1]{\textcolor[rgb]{1.00,0.00,0.00}{\textbf{{#1}}}}
    \newcommand{\FunctionTok}[1]{\textcolor[rgb]{0.02,0.16,0.49}{{#1}}}
    \newcommand{\RegionMarkerTok}[1]{{#1}}
    \newcommand{\ErrorTok}[1]{\textcolor[rgb]{1.00,0.00,0.00}{\textbf{{#1}}}}
    \newcommand{\NormalTok}[1]{{#1}}
    
    % Additional commands for more recent versions of Pandoc
    \newcommand{\ConstantTok}[1]{\textcolor[rgb]{0.53,0.00,0.00}{{#1}}}
    \newcommand{\SpecialCharTok}[1]{\textcolor[rgb]{0.25,0.44,0.63}{{#1}}}
    \newcommand{\VerbatimStringTok}[1]{\textcolor[rgb]{0.25,0.44,0.63}{{#1}}}
    \newcommand{\SpecialStringTok}[1]{\textcolor[rgb]{0.73,0.40,0.53}{{#1}}}
    \newcommand{\ImportTok}[1]{{#1}}
    \newcommand{\DocumentationTok}[1]{\textcolor[rgb]{0.73,0.13,0.13}{\textit{{#1}}}}
    \newcommand{\AnnotationTok}[1]{\textcolor[rgb]{0.38,0.63,0.69}{\textbf{\textit{{#1}}}}}
    \newcommand{\CommentVarTok}[1]{\textcolor[rgb]{0.38,0.63,0.69}{\textbf{\textit{{#1}}}}}
    \newcommand{\VariableTok}[1]{\textcolor[rgb]{0.10,0.09,0.49}{{#1}}}
    \newcommand{\ControlFlowTok}[1]{\textcolor[rgb]{0.00,0.44,0.13}{\textbf{{#1}}}}
    \newcommand{\OperatorTok}[1]{\textcolor[rgb]{0.40,0.40,0.40}{{#1}}}
    \newcommand{\BuiltInTok}[1]{{#1}}
    \newcommand{\ExtensionTok}[1]{{#1}}
    \newcommand{\PreprocessorTok}[1]{\textcolor[rgb]{0.74,0.48,0.00}{{#1}}}
    \newcommand{\AttributeTok}[1]{\textcolor[rgb]{0.49,0.56,0.16}{{#1}}}
    \newcommand{\InformationTok}[1]{\textcolor[rgb]{0.38,0.63,0.69}{\textbf{\textit{{#1}}}}}
    \newcommand{\WarningTok}[1]{\textcolor[rgb]{0.38,0.63,0.69}{\textbf{\textit{{#1}}}}}
    
    
    % Define a nice break command that doesn't care if a line doesn't already
    % exist.
    \def\br{\hspace*{\fill} \\* }
    % Math Jax compatability definitions
    \def\gt{>}
    \def\lt{<}
    % Document parameters
    \title{GraphIntro}
    
    
    

    % Pygments definitions
    
\makeatletter
\def\PY@reset{\let\PY@it=\relax \let\PY@bf=\relax%
    \let\PY@ul=\relax \let\PY@tc=\relax%
    \let\PY@bc=\relax \let\PY@ff=\relax}
\def\PY@tok#1{\csname PY@tok@#1\endcsname}
\def\PY@toks#1+{\ifx\relax#1\empty\else%
    \PY@tok{#1}\expandafter\PY@toks\fi}
\def\PY@do#1{\PY@bc{\PY@tc{\PY@ul{%
    \PY@it{\PY@bf{\PY@ff{#1}}}}}}}
\def\PY#1#2{\PY@reset\PY@toks#1+\relax+\PY@do{#2}}

\expandafter\def\csname PY@tok@nc\endcsname{\let\PY@bf=\textbf\def\PY@tc##1{\textcolor[rgb]{0.00,0.00,1.00}{##1}}}
\expandafter\def\csname PY@tok@kc\endcsname{\let\PY@bf=\textbf\def\PY@tc##1{\textcolor[rgb]{0.00,0.50,0.00}{##1}}}
\expandafter\def\csname PY@tok@vg\endcsname{\def\PY@tc##1{\textcolor[rgb]{0.10,0.09,0.49}{##1}}}
\expandafter\def\csname PY@tok@si\endcsname{\let\PY@bf=\textbf\def\PY@tc##1{\textcolor[rgb]{0.73,0.40,0.53}{##1}}}
\expandafter\def\csname PY@tok@s\endcsname{\def\PY@tc##1{\textcolor[rgb]{0.73,0.13,0.13}{##1}}}
\expandafter\def\csname PY@tok@nb\endcsname{\def\PY@tc##1{\textcolor[rgb]{0.00,0.50,0.00}{##1}}}
\expandafter\def\csname PY@tok@cs\endcsname{\let\PY@it=\textit\def\PY@tc##1{\textcolor[rgb]{0.25,0.50,0.50}{##1}}}
\expandafter\def\csname PY@tok@go\endcsname{\def\PY@tc##1{\textcolor[rgb]{0.53,0.53,0.53}{##1}}}
\expandafter\def\csname PY@tok@gu\endcsname{\let\PY@bf=\textbf\def\PY@tc##1{\textcolor[rgb]{0.50,0.00,0.50}{##1}}}
\expandafter\def\csname PY@tok@mi\endcsname{\def\PY@tc##1{\textcolor[rgb]{0.40,0.40,0.40}{##1}}}
\expandafter\def\csname PY@tok@sa\endcsname{\def\PY@tc##1{\textcolor[rgb]{0.73,0.13,0.13}{##1}}}
\expandafter\def\csname PY@tok@no\endcsname{\def\PY@tc##1{\textcolor[rgb]{0.53,0.00,0.00}{##1}}}
\expandafter\def\csname PY@tok@nd\endcsname{\def\PY@tc##1{\textcolor[rgb]{0.67,0.13,1.00}{##1}}}
\expandafter\def\csname PY@tok@fm\endcsname{\def\PY@tc##1{\textcolor[rgb]{0.00,0.00,1.00}{##1}}}
\expandafter\def\csname PY@tok@nl\endcsname{\def\PY@tc##1{\textcolor[rgb]{0.63,0.63,0.00}{##1}}}
\expandafter\def\csname PY@tok@mf\endcsname{\def\PY@tc##1{\textcolor[rgb]{0.40,0.40,0.40}{##1}}}
\expandafter\def\csname PY@tok@ss\endcsname{\def\PY@tc##1{\textcolor[rgb]{0.10,0.09,0.49}{##1}}}
\expandafter\def\csname PY@tok@gt\endcsname{\def\PY@tc##1{\textcolor[rgb]{0.00,0.27,0.87}{##1}}}
\expandafter\def\csname PY@tok@s2\endcsname{\def\PY@tc##1{\textcolor[rgb]{0.73,0.13,0.13}{##1}}}
\expandafter\def\csname PY@tok@kd\endcsname{\let\PY@bf=\textbf\def\PY@tc##1{\textcolor[rgb]{0.00,0.50,0.00}{##1}}}
\expandafter\def\csname PY@tok@kp\endcsname{\def\PY@tc##1{\textcolor[rgb]{0.00,0.50,0.00}{##1}}}
\expandafter\def\csname PY@tok@kt\endcsname{\def\PY@tc##1{\textcolor[rgb]{0.69,0.00,0.25}{##1}}}
\expandafter\def\csname PY@tok@gi\endcsname{\def\PY@tc##1{\textcolor[rgb]{0.00,0.63,0.00}{##1}}}
\expandafter\def\csname PY@tok@gd\endcsname{\def\PY@tc##1{\textcolor[rgb]{0.63,0.00,0.00}{##1}}}
\expandafter\def\csname PY@tok@vc\endcsname{\def\PY@tc##1{\textcolor[rgb]{0.10,0.09,0.49}{##1}}}
\expandafter\def\csname PY@tok@sb\endcsname{\def\PY@tc##1{\textcolor[rgb]{0.73,0.13,0.13}{##1}}}
\expandafter\def\csname PY@tok@s1\endcsname{\def\PY@tc##1{\textcolor[rgb]{0.73,0.13,0.13}{##1}}}
\expandafter\def\csname PY@tok@mh\endcsname{\def\PY@tc##1{\textcolor[rgb]{0.40,0.40,0.40}{##1}}}
\expandafter\def\csname PY@tok@w\endcsname{\def\PY@tc##1{\textcolor[rgb]{0.73,0.73,0.73}{##1}}}
\expandafter\def\csname PY@tok@o\endcsname{\def\PY@tc##1{\textcolor[rgb]{0.40,0.40,0.40}{##1}}}
\expandafter\def\csname PY@tok@na\endcsname{\def\PY@tc##1{\textcolor[rgb]{0.49,0.56,0.16}{##1}}}
\expandafter\def\csname PY@tok@ge\endcsname{\let\PY@it=\textit}
\expandafter\def\csname PY@tok@gs\endcsname{\let\PY@bf=\textbf}
\expandafter\def\csname PY@tok@c\endcsname{\let\PY@it=\textit\def\PY@tc##1{\textcolor[rgb]{0.25,0.50,0.50}{##1}}}
\expandafter\def\csname PY@tok@k\endcsname{\let\PY@bf=\textbf\def\PY@tc##1{\textcolor[rgb]{0.00,0.50,0.00}{##1}}}
\expandafter\def\csname PY@tok@ne\endcsname{\let\PY@bf=\textbf\def\PY@tc##1{\textcolor[rgb]{0.82,0.25,0.23}{##1}}}
\expandafter\def\csname PY@tok@il\endcsname{\def\PY@tc##1{\textcolor[rgb]{0.40,0.40,0.40}{##1}}}
\expandafter\def\csname PY@tok@mb\endcsname{\def\PY@tc##1{\textcolor[rgb]{0.40,0.40,0.40}{##1}}}
\expandafter\def\csname PY@tok@gp\endcsname{\let\PY@bf=\textbf\def\PY@tc##1{\textcolor[rgb]{0.00,0.00,0.50}{##1}}}
\expandafter\def\csname PY@tok@sx\endcsname{\def\PY@tc##1{\textcolor[rgb]{0.00,0.50,0.00}{##1}}}
\expandafter\def\csname PY@tok@gh\endcsname{\let\PY@bf=\textbf\def\PY@tc##1{\textcolor[rgb]{0.00,0.00,0.50}{##1}}}
\expandafter\def\csname PY@tok@sd\endcsname{\let\PY@it=\textit\def\PY@tc##1{\textcolor[rgb]{0.73,0.13,0.13}{##1}}}
\expandafter\def\csname PY@tok@dl\endcsname{\def\PY@tc##1{\textcolor[rgb]{0.73,0.13,0.13}{##1}}}
\expandafter\def\csname PY@tok@cpf\endcsname{\let\PY@it=\textit\def\PY@tc##1{\textcolor[rgb]{0.25,0.50,0.50}{##1}}}
\expandafter\def\csname PY@tok@ni\endcsname{\let\PY@bf=\textbf\def\PY@tc##1{\textcolor[rgb]{0.60,0.60,0.60}{##1}}}
\expandafter\def\csname PY@tok@sc\endcsname{\def\PY@tc##1{\textcolor[rgb]{0.73,0.13,0.13}{##1}}}
\expandafter\def\csname PY@tok@vi\endcsname{\def\PY@tc##1{\textcolor[rgb]{0.10,0.09,0.49}{##1}}}
\expandafter\def\csname PY@tok@nf\endcsname{\def\PY@tc##1{\textcolor[rgb]{0.00,0.00,1.00}{##1}}}
\expandafter\def\csname PY@tok@gr\endcsname{\def\PY@tc##1{\textcolor[rgb]{1.00,0.00,0.00}{##1}}}
\expandafter\def\csname PY@tok@ow\endcsname{\let\PY@bf=\textbf\def\PY@tc##1{\textcolor[rgb]{0.67,0.13,1.00}{##1}}}
\expandafter\def\csname PY@tok@sr\endcsname{\def\PY@tc##1{\textcolor[rgb]{0.73,0.40,0.53}{##1}}}
\expandafter\def\csname PY@tok@ch\endcsname{\let\PY@it=\textit\def\PY@tc##1{\textcolor[rgb]{0.25,0.50,0.50}{##1}}}
\expandafter\def\csname PY@tok@nv\endcsname{\def\PY@tc##1{\textcolor[rgb]{0.10,0.09,0.49}{##1}}}
\expandafter\def\csname PY@tok@c1\endcsname{\let\PY@it=\textit\def\PY@tc##1{\textcolor[rgb]{0.25,0.50,0.50}{##1}}}
\expandafter\def\csname PY@tok@err\endcsname{\def\PY@bc##1{\setlength{\fboxsep}{0pt}\fcolorbox[rgb]{1.00,0.00,0.00}{1,1,1}{\strut ##1}}}
\expandafter\def\csname PY@tok@cp\endcsname{\def\PY@tc##1{\textcolor[rgb]{0.74,0.48,0.00}{##1}}}
\expandafter\def\csname PY@tok@vm\endcsname{\def\PY@tc##1{\textcolor[rgb]{0.10,0.09,0.49}{##1}}}
\expandafter\def\csname PY@tok@se\endcsname{\let\PY@bf=\textbf\def\PY@tc##1{\textcolor[rgb]{0.73,0.40,0.13}{##1}}}
\expandafter\def\csname PY@tok@nn\endcsname{\let\PY@bf=\textbf\def\PY@tc##1{\textcolor[rgb]{0.00,0.00,1.00}{##1}}}
\expandafter\def\csname PY@tok@mo\endcsname{\def\PY@tc##1{\textcolor[rgb]{0.40,0.40,0.40}{##1}}}
\expandafter\def\csname PY@tok@sh\endcsname{\def\PY@tc##1{\textcolor[rgb]{0.73,0.13,0.13}{##1}}}
\expandafter\def\csname PY@tok@kn\endcsname{\let\PY@bf=\textbf\def\PY@tc##1{\textcolor[rgb]{0.00,0.50,0.00}{##1}}}
\expandafter\def\csname PY@tok@kr\endcsname{\let\PY@bf=\textbf\def\PY@tc##1{\textcolor[rgb]{0.00,0.50,0.00}{##1}}}
\expandafter\def\csname PY@tok@nt\endcsname{\let\PY@bf=\textbf\def\PY@tc##1{\textcolor[rgb]{0.00,0.50,0.00}{##1}}}
\expandafter\def\csname PY@tok@m\endcsname{\def\PY@tc##1{\textcolor[rgb]{0.40,0.40,0.40}{##1}}}
\expandafter\def\csname PY@tok@cm\endcsname{\let\PY@it=\textit\def\PY@tc##1{\textcolor[rgb]{0.25,0.50,0.50}{##1}}}
\expandafter\def\csname PY@tok@bp\endcsname{\def\PY@tc##1{\textcolor[rgb]{0.00,0.50,0.00}{##1}}}

\def\PYZbs{\char`\\}
\def\PYZus{\char`\_}
\def\PYZob{\char`\{}
\def\PYZcb{\char`\}}
\def\PYZca{\char`\^}
\def\PYZam{\char`\&}
\def\PYZlt{\char`\<}
\def\PYZgt{\char`\>}
\def\PYZsh{\char`\#}
\def\PYZpc{\char`\%}
\def\PYZdl{\char`\$}
\def\PYZhy{\char`\-}
\def\PYZsq{\char`\'}
\def\PYZdq{\char`\"}
\def\PYZti{\char`\~}
% for compatibility with earlier versions
\def\PYZat{@}
\def\PYZlb{[}
\def\PYZrb{]}
\makeatother


    % Exact colors from NB
    \definecolor{incolor}{rgb}{0.0, 0.0, 0.5}
    \definecolor{outcolor}{rgb}{0.545, 0.0, 0.0}



    
    % Prevent overflowing lines due to hard-to-break entities
    \sloppy 
    % Setup hyperref package
    \hypersetup{
      breaklinks=true,  % so long urls are correctly broken across lines
      colorlinks=true,
      urlcolor=urlcolor,
      linkcolor=linkcolor,
      citecolor=citecolor,
      }
    % Slightly bigger margins than the latex defaults
    
    \geometry{verbose,tmargin=1in,bmargin=1in,lmargin=1in,rmargin=1in}
    
    

    \begin{document}
    
    
    \maketitle
    
    

    
    \section{Introducción de Grafos
Dinámicos}\label{introducciuxf3n-de-grafos-dinuxe1micos}

    \subsection{Crear un grafo}\label{crear-un-grafo}

    Para crear un grafo, instanciamos un objeto de \textbf{Graph} y pasamos
como parámetros lo siguinte:

\begin{verbatim}
- sources ([int]): una lista de nodos fuentes
- targets ([int]): una lista de nodos destinos 
- weights ([int]): una lista de sus respectivos pesos.
\end{verbatim}

    \begin{Verbatim}[commandchars=\\\{\}]
{\color{incolor}In [{\color{incolor}1}]:} \PY{k+kn}{from} \PY{n+nn}{graph}\PY{n+nn}{.}\PY{n+nn}{Graph} \PY{k}{import} \PY{n}{Graph} 
        
        \PY{n}{sources} \PY{o}{=} \PY{p}{[}\PY{l+m+mi}{1}\PY{p}{,} \PY{l+m+mi}{2}\PY{p}{,} \PY{l+m+mi}{3}\PY{p}{,} \PY{l+m+mi}{0}\PY{p}{,} \PY{l+m+mi}{5}\PY{p}{,} \PY{l+m+mi}{1}\PY{p}{,} \PY{l+m+mi}{1}\PY{p}{,} \PY{l+m+mi}{3}\PY{p}{]}
        \PY{n}{targets} \PY{o}{=} \PY{p}{[}\PY{l+m+mi}{0}\PY{p}{,} \PY{l+m+mi}{0}\PY{p}{,} \PY{l+m+mi}{0}\PY{p}{,} \PY{l+m+mi}{5}\PY{p}{,} \PY{l+m+mi}{0}\PY{p}{,} \PY{l+m+mi}{2}\PY{p}{,} \PY{l+m+mi}{3}\PY{p}{,} \PY{l+m+mi}{1}\PY{p}{]}
        \PY{n}{weights} \PY{o}{=} \PY{p}{[}\PY{l+m+mi}{2}\PY{p}{,} \PY{l+m+mi}{2}\PY{p}{,} \PY{l+m+mi}{4}\PY{p}{,} \PY{l+m+mi}{1}\PY{p}{,} \PY{l+m+mi}{1}\PY{p}{,} \PY{l+m+mi}{3}\PY{p}{,} \PY{l+m+mi}{2}\PY{p}{,} \PY{l+m+mi}{2}\PY{p}{]}
        
        \PY{n}{graph} \PY{o}{=} \PY{n}{Graph}\PY{p}{(}\PY{n}{sources}\PY{p}{,} \PY{n}{targets}\PY{p}{,} \PY{n}{weights}\PY{p}{)}
        
        \PY{n}{graph}\PY{o}{.}\PY{n}{print\PYZus{}r}\PY{p}{(}\PY{p}{)}
\end{Verbatim}


    \begin{Verbatim}[commandchars=\\\{\}]
Source:  [1 2 3 0 5 1 1 3]
Target:  [0 0 0 5 0 2 3 1]
Weight:  [2 2 4 1 1 3 2 2]
Vertex:  [0 1 2 3 5]

    \end{Verbatim}

    \subsection{Graficar}\label{graficar}

    Para graficar el grafo, la librería cuenta con una implementación de
networkx. Solo falta usar el comando \textbf{draw}:

\begin{verbatim}
with_weight (boolean)[default=True][opcional]: Si muestra los pesos de cada arco (edge)
\end{verbatim}

    \begin{Verbatim}[commandchars=\\\{\}]
{\color{incolor}In [{\color{incolor}2}]:} \PY{n}{graph}\PY{o}{.}\PY{n}{draw}\PY{p}{(}\PY{n}{with\PYZus{}weight}\PY{o}{=}\PY{k+kc}{False}\PY{p}{)}
\end{Verbatim}


    \begin{center}
    \adjustimage{max size={0.9\linewidth}{0.9\paperheight}}{output_6_0.png}
    \end{center}
    { \hspace*{\fill} \\}
    
    \subsection{Métodos dinámicos}\label{muxe9todos-dinuxe1micos}

    Los grafos dinámicos cuentan con un conjunto de métodos para modificar
la estructura del propio grafo, la mayoría posee una visualización
gráfica.

    \subsubsection{Insertar un Arco (Incremental
Edge)}\label{insertar-un-arco-incremental-edge}

    Para agregar un arco al grafo utilizamos el método
\textbf{dynamic\_incremental\_edge} con los siguientes parámetros:

\begin{verbatim}
source (int): Nodo fuente
target (int): Nodo destino
weight (int) [opcional]: Peso del arco
\end{verbatim}

    \begin{Verbatim}[commandchars=\\\{\}]
{\color{incolor}In [{\color{incolor}3}]:} \PY{n}{graph}\PY{o}{.}\PY{n}{dynamic\PYZus{}incremental\PYZus{}edge}\PY{p}{(}\PY{n}{source}\PY{o}{=}\PY{l+m+mi}{3}\PY{p}{,} \PY{n}{target}\PY{o}{=}\PY{l+m+mi}{5}\PY{p}{)}
        \PY{n}{graph}\PY{o}{.}\PY{n}{draw}\PY{p}{(}\PY{n}{with\PYZus{}weight}\PY{o}{=}\PY{k+kc}{False}\PY{p}{)}
\end{Verbatim}


    \begin{center}
    \adjustimage{max size={0.9\linewidth}{0.9\paperheight}}{output_11_0.png}
    \end{center}
    { \hspace*{\fill} \\}
    
    \subsubsection{Actualizar un arco (Update
Edge)}\label{actualizar-un-arco-update-edge}

    Para actualizar un arco por su peso utilizamos el método
\textbf{nodes\_update} con los siguientes parámetros:

\begin{verbatim}
source (int): Nodo fuente del arco a actualizar
target (int): Nodo destino del arco a actualizar
weight (int) [default=1] [optional]: Peso del arco a actualizar
\end{verbatim}

    \begin{Verbatim}[commandchars=\\\{\}]
{\color{incolor}In [{\color{incolor}4}]:} \PY{n}{graph}\PY{o}{.}\PY{n}{nodes\PYZus{}update}\PY{p}{(}\PY{n}{source}\PY{o}{=}\PY{l+m+mi}{3}\PY{p}{,} \PY{n}{target}\PY{o}{=}\PY{l+m+mi}{0}\PY{p}{)}
        \PY{n}{graph}\PY{o}{.}\PY{n}{draw}\PY{p}{(}\PY{p}{)}
\end{Verbatim}


    \begin{center}
    \adjustimage{max size={0.9\linewidth}{0.9\paperheight}}{output_14_0.png}
    \end{center}
    { \hspace*{\fill} \\}
    
    \subsubsection{Eliminar un arco (Remove
edge)}\label{eliminar-un-arco-remove-edge}

    Para eliminar un arco del grafo utilizamos el método
\textbf{dynamic\_decreasing\_edge} con los siguientes parámetros:

\begin{verbatim}
source (int): Nodo fuente del arco a eliminar
target (int): Nodo destino del arco a eliminar
\end{verbatim}

    \begin{Verbatim}[commandchars=\\\{\}]
{\color{incolor}In [{\color{incolor}5}]:} \PY{n}{graph}\PY{o}{.}\PY{n}{dynamic\PYZus{}decreasing\PYZus{}edge}\PY{p}{(}\PY{n}{source}\PY{o}{=}\PY{l+m+mi}{1}\PY{p}{,} \PY{n}{target}\PY{o}{=}\PY{l+m+mi}{2}\PY{p}{)}
        \PY{n}{graph}\PY{o}{.}\PY{n}{draw}\PY{p}{(}\PY{n}{with\PYZus{}weight}\PY{o}{=}\PY{k+kc}{False}\PY{p}{)}
\end{Verbatim}


    \begin{center}
    \adjustimage{max size={0.9\linewidth}{0.9\paperheight}}{output_17_0.png}
    \end{center}
    { \hspace*{\fill} \\}
    
    \subsubsection{Insertar un nodo (Incremental
nodes)}\label{insertar-un-nodo-incremental-nodes}

    Para insertar un nodo al grafo utilizamos el método
\textbf{dynamic\_incremental\_node} con los siguientes parámetros:

\begin{verbatim}
node (int): Nodo a agregar

sources ([int]): Lista de nodos fuentes del grafo al nuevo nodo
w_sources ([int]): Pesos de los arcos de la lista de nodos fuentes

targets ([int]): Lista de nodos destinos del grado desde el nuevo nodo
w_targets ([int]): Pesos de los arcos de la lista de nodos objetivos
\end{verbatim}

    \begin{Verbatim}[commandchars=\\\{\}]
{\color{incolor}In [{\color{incolor}6}]:} \PY{n}{graph}\PY{o}{.}\PY{n}{print\PYZus{}r}\PY{p}{(}\PY{p}{)}
        \PY{n}{graph}\PY{o}{.}\PY{n}{dynamic\PYZus{}incremental\PYZus{}node}\PY{p}{(}\PY{n}{node}\PY{o}{=}\PY{l+m+mi}{6}\PY{p}{,} \PY{n}{sources}\PY{o}{=}\PY{p}{[}\PY{l+m+mi}{2}\PY{p}{,} \PY{l+m+mi}{0}\PY{p}{,} \PY{l+m+mi}{5}\PY{p}{]}\PY{p}{,} \PY{n}{w\PYZus{}sources}\PY{o}{=}\PY{p}{[}\PY{l+m+mi}{2}\PY{p}{,} \PY{l+m+mi}{4}\PY{p}{,} \PY{l+m+mi}{5}\PY{p}{]}\PY{p}{,} \PY{n}{targets}\PY{o}{=}\PY{p}{[}\PY{l+m+mi}{5}\PY{p}{]}\PY{p}{,} \PY{n}{w\PYZus{}targets}\PY{o}{=}\PY{p}{[}\PY{l+m+mi}{2}\PY{p}{]}\PY{p}{)}
        \PY{n+nb}{print}\PY{p}{(}\PY{l+s+s1}{\PYZsq{}}\PY{l+s+s1}{\PYZhy{}\PYZhy{}\PYZhy{}Incremental nodes\PYZhy{}\PYZhy{}\PYZhy{}}\PY{l+s+s1}{\PYZsq{}}\PY{p}{)}
        \PY{n}{graph}\PY{o}{.}\PY{n}{print\PYZus{}r}\PY{p}{(}\PY{p}{)}
        \PY{n}{graph}\PY{o}{.}\PY{n}{draw}\PY{p}{(}\PY{p}{)}
\end{Verbatim}


    \begin{Verbatim}[commandchars=\\\{\}]
Source:  [1 2 3 0 5 1 3 3]
Target:  [0 0 0 5 0 3 1 5]
Weight:  [2 2 1 1 1 2 2 1]
Vertex:  [0 1 2 3 5]
---Incremental nodes---
Source:  [1 2 3 0 5 1 3 3 2 0 5 6]
Target:  [0 0 0 5 0 3 1 5 6 6 6 5]
Weight:  [2 2 1 1 1 2 2 1 2 4 5 2]
Vertex:  [0 1 2 3 5 6]

    \end{Verbatim}

    \begin{center}
    \adjustimage{max size={0.9\linewidth}{0.9\paperheight}}{output_20_1.png}
    \end{center}
    { \hspace*{\fill} \\}
    
    \subsubsection{Eliminar un nodo (Decreasing
nodes)}\label{eliminar-un-nodo-decreasing-nodes}

    Para elimnar un nodo al grafo utilizamos el método
\textbf{dynamic\_decreasing\_node} con los siguientes parámetros:

\begin{verbatim}
node (int): Nodo a eliminar
\end{verbatim}

    \begin{Verbatim}[commandchars=\\\{\}]
{\color{incolor}In [{\color{incolor}7}]:} \PY{n}{graph}\PY{o}{.}\PY{n}{print\PYZus{}r}\PY{p}{(}\PY{p}{)}
        \PY{n}{graph}\PY{o}{.}\PY{n}{dynamic\PYZus{}decreasing\PYZus{}node}\PY{p}{(}\PY{n}{node}\PY{o}{=}\PY{l+m+mi}{5}\PY{p}{)}
        \PY{n}{graph}\PY{o}{.}\PY{n}{draw}\PY{p}{(}\PY{p}{)}
\end{Verbatim}


    \begin{Verbatim}[commandchars=\\\{\}]
Source:  [1 2 3 0 5 1 3 3 2 0 5 6]
Target:  [0 0 0 5 0 3 1 5 6 6 6 5]
Weight:  [2 2 1 1 1 2 2 1 2 4 5 2]
Vertex:  [0 1 2 3 5 6]

    \end{Verbatim}

    \begin{center}
    \adjustimage{max size={0.9\linewidth}{0.9\paperheight}}{output_23_1.png}
    \end{center}
    { \hspace*{\fill} \\}
    
    \subsection{Grafo aleatorio}\label{grafo-aleatorio}

    \subsubsection{Crear grafo aleatorio}\label{crear-grafo-aleatorio}

    Para crear un grafo con parámetros aleatorios usamos el médoto
\textbf{creategraph} con los siguientes parámetros:

\begin{verbatim}
total_nodes (int): Número de nodos que tendrá el grafo
pro_edges (float [0 <-> 1]): Probabilidad que entre nodos exista un arco, entre uno y cero.
weights ([int]) [opcional]: Lista de los pesos para asignar aleatoriamente a los arcos. Por defecto de uno a nueve.
directed (boolean) [opcional]: Si el grafo es dirigido. Por defecto True.
\end{verbatim}

    \begin{Verbatim}[commandchars=\\\{\}]
{\color{incolor}In [{\color{incolor}8}]:} \PY{n}{random\PYZus{}graph} \PY{o}{=} \PY{n}{Graph}\PY{o}{.}\PY{n}{creategraph}\PY{p}{(}\PY{n}{total\PYZus{}nodes}\PY{o}{=}\PY{l+m+mi}{5}\PY{p}{,} \PY{n}{pro\PYZus{}edges}\PY{o}{=}\PY{l+m+mf}{0.85}\PY{p}{)}
        \PY{n}{random\PYZus{}graph}\PY{o}{.}\PY{n}{print\PYZus{}r}\PY{p}{(}\PY{p}{)}
        \PY{n}{random\PYZus{}graph}\PY{o}{.}\PY{n}{draw}\PY{p}{(}\PY{p}{)}
\end{Verbatim}


    \begin{Verbatim}[commandchars=\\\{\}]
Source:  [0. 0. 0. 1. 1. 2. 2. 3.]
Target:  [1. 2. 4. 2. 4. 3. 4. 4.]
Weight:  [2. 9. 8. 2. 1. 7. 6. 6.]
Vertex:  [0. 1. 2. 3. 4.]

    \end{Verbatim}

    \begin{center}
    \adjustimage{max size={0.9\linewidth}{0.9\paperheight}}{output_27_1.png}
    \end{center}
    { \hspace*{\fill} \\}
    
    \subsubsection{Insertar un arco aleatorio (random insert
edge)}\label{insertar-un-arco-aleatorio-random-insert-edge}

    Para insertar un arco aleatorio al grafo utilizamos el método
\textbf{dynamic\_incremental\_random\_edge} con los siguientes
parámetros:

\begin{verbatim}
weights ([int]) [opcional]: Lista de los pesos para asignar aleatoriamente a los arcos. Por defecto de uno a nueve
\end{verbatim}

    \begin{Verbatim}[commandchars=\\\{\}]
{\color{incolor}In [{\color{incolor}9}]:} \PY{n}{random\PYZus{}graph}\PY{o}{.}\PY{n}{dynamic\PYZus{}incremental\PYZus{}random\PYZus{}edge}\PY{p}{(}\PY{p}{)}
        \PY{n}{random\PYZus{}graph}\PY{o}{.}\PY{n}{draw}\PY{p}{(}\PY{p}{)}
\end{Verbatim}


    \begin{center}
    \adjustimage{max size={0.9\linewidth}{0.9\paperheight}}{output_30_0.png}
    \end{center}
    { \hspace*{\fill} \\}
    
    \subsubsection{Eliminar un arco aleatorio (random delete
edge)}\label{eliminar-un-arco-aleatorio-random-delete-edge}

    Para eliminar un arco aleatorio del grafo utilizamos el método
\textbf{dynamic\_decreasing\_random\_edge}, sin parámetros.

    \begin{Verbatim}[commandchars=\\\{\}]
{\color{incolor}In [{\color{incolor}10}]:} \PY{n}{random\PYZus{}graph}\PY{o}{.}\PY{n}{dynamic\PYZus{}decreasing\PYZus{}random\PYZus{}edge}\PY{p}{(}\PY{p}{)}
         \PY{n}{random\PYZus{}graph}\PY{o}{.}\PY{n}{draw}\PY{p}{(}\PY{p}{)}
\end{Verbatim}


    \begin{center}
    \adjustimage{max size={0.9\linewidth}{0.9\paperheight}}{output_33_0.png}
    \end{center}
    { \hspace*{\fill} \\}
    
    \subsubsection{Actualizar arco aleatorio (random update
edge)}\label{actualizar-arco-aleatorio-random-update-edge}

    Para actualizar el peso de un arco aleatorio del grafo utilizamos el
método \textbf{nodes\_update\_random}, sin parámetros.

    \begin{Verbatim}[commandchars=\\\{\}]
{\color{incolor}In [{\color{incolor}11}]:} \PY{n}{random\PYZus{}graph}\PY{o}{.}\PY{n}{nodes\PYZus{}update\PYZus{}random}\PY{p}{(}\PY{p}{)}
         \PY{n}{random\PYZus{}graph}\PY{o}{.}\PY{n}{draw}\PY{p}{(}\PY{p}{)}
\end{Verbatim}


    \begin{center}
    \adjustimage{max size={0.9\linewidth}{0.9\paperheight}}{output_36_0.png}
    \end{center}
    { \hspace*{\fill} \\}
    
    \subsubsection{Insertar nodo aleatorio (Random insert
nodes)}\label{insertar-nodo-aleatorio-random-insert-nodes}

    Para insertar un nodo aleatorio al grafo utilizamos el método
\textbf{dynamic\_incremental\_random\_node} con los siguientes
parámetros:

\begin{verbatim}
num_edges (int) [opcional]: Número de arcos que tendrá el nuevo nodo. Por defecto (1)
weights ([int]) [opcional]: Lista de los pesos para asignar aleatoriamente a los arcos. Por defecto de uno a nueve
\end{verbatim}

    \begin{Verbatim}[commandchars=\\\{\}]
{\color{incolor}In [{\color{incolor}12}]:} \PY{n}{random\PYZus{}graph}\PY{o}{.}\PY{n}{dynamic\PYZus{}incremental\PYZus{}random\PYZus{}node}\PY{p}{(}\PY{n}{num\PYZus{}edges}\PY{o}{=}\PY{l+m+mi}{2}\PY{p}{)}
         \PY{n}{random\PYZus{}graph}\PY{o}{.}\PY{n}{draw}\PY{p}{(}\PY{p}{)}
\end{Verbatim}


    \begin{Verbatim}[commandchars=\\\{\}]
5.0

    \end{Verbatim}

    \begin{center}
    \adjustimage{max size={0.9\linewidth}{0.9\paperheight}}{output_39_1.png}
    \end{center}
    { \hspace*{\fill} \\}
    
    \subsubsection{Eliminar nodo aleatorio (random delete
nodes)}\label{eliminar-nodo-aleatorio-random-delete-nodes}

    Para eliminar un nodo aleatorio al grafo utilizamos el método
\textbf{dynamic\_decreasing\_random\_node}, no posee parámetros

    \begin{Verbatim}[commandchars=\\\{\}]
{\color{incolor}In [{\color{incolor}13}]:} \PY{n}{random\PYZus{}graph}\PY{o}{.}\PY{n}{dynamic\PYZus{}decreasing\PYZus{}random\PYZus{}node}\PY{p}{(}\PY{p}{)}
         \PY{n}{random\PYZus{}graph}\PY{o}{.}\PY{n}{draw}\PY{p}{(}\PY{p}{)}
\end{Verbatim}


    \begin{center}
    \adjustimage{max size={0.9\linewidth}{0.9\paperheight}}{output_42_0.png}
    \end{center}
    { \hspace*{\fill} \\}
    

    % Add a bibliography block to the postdoc
    
    
    
    \end{document}
